\documentclass[]{article}
\usepackage{lmodern}
\usepackage{amssymb,amsmath}
\usepackage{ifxetex,ifluatex}
\usepackage{fixltx2e} % provides \textsubscript
\ifnum 0\ifxetex 1\fi\ifluatex 1\fi=0 % if pdftex
  \usepackage[T1]{fontenc}
  \usepackage[utf8]{inputenc}
\else % if luatex or xelatex
  \ifxetex
    \usepackage{mathspec}
  \else
    \usepackage{fontspec}
  \fi
  \defaultfontfeatures{Ligatures=TeX,Scale=MatchLowercase}
\fi
% use upquote if available, for straight quotes in verbatim environments
\IfFileExists{upquote.sty}{\usepackage{upquote}}{}
% use microtype if available
\IfFileExists{microtype.sty}{%
\usepackage{microtype}
\UseMicrotypeSet[protrusion]{basicmath} % disable protrusion for tt fonts
}{}
\usepackage[margin=1in]{geometry}
\usepackage{hyperref}
\hypersetup{unicode=true,
            pdfborder={0 0 0},
            breaklinks=true}
\urlstyle{same}  % don't use monospace font for urls
\usepackage{natbib}
\bibliographystyle{plainnat}
\usepackage{longtable,booktabs}
\usepackage{graphicx,grffile}
\makeatletter
\def\maxwidth{\ifdim\Gin@nat@width>\linewidth\linewidth\else\Gin@nat@width\fi}
\def\maxheight{\ifdim\Gin@nat@height>\textheight\textheight\else\Gin@nat@height\fi}
\makeatother
% Scale images if necessary, so that they will not overflow the page
% margins by default, and it is still possible to overwrite the defaults
% using explicit options in \includegraphics[width, height, ...]{}
\setkeys{Gin}{width=\maxwidth,height=\maxheight,keepaspectratio}
\IfFileExists{parskip.sty}{%
\usepackage{parskip}
}{% else
\setlength{\parindent}{0pt}
\setlength{\parskip}{6pt plus 2pt minus 1pt}
}
\setlength{\emergencystretch}{3em}  % prevent overfull lines
\providecommand{\tightlist}{%
  \setlength{\itemsep}{0pt}\setlength{\parskip}{0pt}}
\setcounter{secnumdepth}{5}
% Redefines (sub)paragraphs to behave more like sections
\ifx\paragraph\undefined\else
\let\oldparagraph\paragraph
\renewcommand{\paragraph}[1]{\oldparagraph{#1}\mbox{}}
\fi
\ifx\subparagraph\undefined\else
\let\oldsubparagraph\subparagraph
\renewcommand{\subparagraph}[1]{\oldsubparagraph{#1}\mbox{}}
\fi

%%% Use protect on footnotes to avoid problems with footnotes in titles
\let\rmarkdownfootnote\footnote%
\def\footnote{\protect\rmarkdownfootnote}

%%% Change title format to be more compact
\usepackage{titling}

% Create subtitle command for use in maketitle
\newcommand{\subtitle}[1]{
  \posttitle{
    \begin{center}\large#1\end{center}
    }
}

\setlength{\droptitle}{-2em}
  \title{}
  \pretitle{\vspace{\droptitle}}
  \posttitle{}
  \author{}
  \preauthor{}\postauthor{}
  \date{}
  \predate{}\postdate{}

\usepackage{booktabs}
\usepackage[english, ngerman]{babel}

\begin{document}

{
\setcounter{tocdepth}{3}
\tableofcontents
}
\section{Konzipieren einer Arbeit}\label{konzipieren-einer-arbeit}

\begin{figure}

{\centering \includegraphics{images/konzipieren} 

}

\caption{Konzipieren (Übersicht)}\label{fig:unnamed-chunk-2}
\end{figure}

\subsection{Von der Idee zum Konzept}\label{von-der-idee-zum-konzept}

Am Anfang jeder Arbeit sollte ein eigenes -- wenn auch noch vages --
Interesse an einem Problem, an einem Sachgebiet stehen. Persönliches
Engagement ist der beste Start für die harte Arbeit, die vor Ihnen
liegt. Sich eine Thematik von der Betreuerin oder der Professorin bloß
zuweisen zu lassen, ohne eigenes Interesse oder zumindest einen eigenen
Gesichtspunkt einzubringen, ist zwar bequem und spart anfangs auch etwas
Zeit, kann aber leicht in eine Sackgasse führen. Wissenschaftliches
Arbeiten lässt sich nicht mit „Dienst nach Vorschrift`` abarbeiten,
umsetzen oder aussitzen, sondern braucht Motivation, Engagement und
Kreativität.

Eine erste Idee oder vages Interesse ist aber noch lange kein
bearbeitbares Thema. Das Thema muss erst mühsam daraus entwickelt
werden. Stürzt man sich gleich, bloß mit einer vagen Ausgangsüberlegung,
in die Arbeit, dann ist die Gefahr groß, dass das Thema viel zu breit
und allgemein angelegt ist: man stößt auf immer mehr interessante
Facetten, wirft immer neue Fragen auf, und weiß schließlich nicht mehr,
wo anfangen und wo aufhören. Schon die Materialsuche will kein Ende mehr
nehmen, Lesen oder Exzerpieren wird zu einer Lebensaufgabe.

Es zahlt sich daher aus, für die Phase der Konzeptentwicklung genügend
Zeit vorzusehen. Es wird kaum verlorene Zeit sein, denn erstens kann man
sich späteren unnötigen Aufwand ersparen, und zweitens ist das meiste
des hier Gesammelten und Überlegten für die Arbeit bereits verwendbar.
Selbst wenn man durch die Vorarbeiten entdeckt, dass die ursprüngliche
Idee ungeeignet ist und verworfen werden muss, so ist es immer noch
besser, jetzt nochmals neu anzufangen als vielleicht erst Monate später.

Mit diesem Kapitel möchten wir versuchen, Ihnen Tipps und Hilfen für den
Weg von der ersten Idee über die Themenfindung und -eingrenzung bis zum
Exposé der Arbeit zu geben. Leider können das nur relativ allgemeine
Vorschläge sein. Warum? Konzeptionelle Arbeit erfordert Kreativität und
lässt sich mit formalen Verhaltensregeln nicht vollständig beschreiben.
Jeder soll und muss die eigenen Vorlieben und Methoden entdecken und
nutzen. Wie diese individuellen Wege zu einem konkreten Konzept oder
Exposé aber auch aussehen mögen, irgendeine Art von methodischem
Vorgehen ist notwendig, um sich den Rahmen für eine machbare,
überschaubare Arbeit zu schaffen.

\begin{figure}

{\centering \includegraphics{images/von-idee-zum-konzept} 

}

\caption{Von der Idee zum Arbeitsbeginn}\label{fig:unnamed-chunk-3}
\end{figure}

\subsection{Rahmenbedingungen klären}\label{rahmenbedingungen-klaren}

Als Rahmenbedingungen gelten jene äußeren Faktoren, unter denen die
wissenschaftlichen Arbeiten durchgeführt werden müssen. Neben den drei
wichtigsten Faktoren Zeit, Geld und Infrastruktur können auch andere
Bedingungen (z. B. formale Vorschriften) je nach Thema Einfluss auf die
Arbeitsbedingungen haben. Die Untersuchung der äußeren
Rahmenbedingungen, unter denen die Arbeit gemacht werden muss, ist
weniger eine eigene, getrennte Phase in der Projektvorbereitung als eine
Reihe von Fragen, die man sich von Anfang an stellt und die immer wieder
bei der Entwicklung des Konzepts eine Rolle spielen.

Der wichtigste, aber auch am schwersten einzuschätzende Faktor ist die
Zeit. Vorgehensweise, Methodik und die gewählten Arbeitsschritte hängen
natürlich weitgehend vom Typus der Arbeit, dem Umfang und der
Komplexität der Fragestellung ab. Das ist ein weiterer Grund, warum es
schwierig ist, allgemeine Hilfestellungen zu geben. Doch zumindest
können für Qualifikationsarbeiten, wie sie für Studium und
wissenschaftliche Karriere verlangt werden, Richtwerte (z. B. die im
Studienplan vorgesehenen Zeiten) oder durchschnittliche Erfahrungswerte
angegeben werden. Diese Werte dienen als erste Orientierung, von denen
dann auf die entsprechenden Arbeitsphasen rückgerechnet werden kann.

Ein anderer Begrenzungsfaktor ist der bei bestimmten Arbeiten
vorgegebene Abgabetermin. Für Zeitschriftenartikel oder Kongressbeiträge
gibt es fixe Termine, aber auch für Diplomarbeiten kann es eine
festgelegte Bearbeitungsfrist geben. Hier wird man vom Endtermin
ausgehen und rückrechnen, um herauszufinden, wieviel Zeit man sich für
die einzelnen Phasen erlauben kann. Dabei sollte man unterscheiden
zwischen „Brutto`` (die Wochen, Monate \ldots{} bis zur Fertigstellung)
und „Netto`` (die tatsächlich während eines Tages, einer Woche etc. für
die Arbeit verfügbare Zeit).

Die „Nettozeit`` können Sie herausfinden, indem Sie eine Woche hindurch
über die Verwendung Ihrer Zeit Buch führen. Wenn Sie dabei realistische
Zeiten für Essen, Schlafen, Erholung, Fahrten, andere Arbeiten usw.
berücksichtigen und dazu noch Pufferzeiten für unvorhergesehene
Ereignisse mitrechnen (und nicht zu knapp), dann kommen Sie
wahrscheinlich zu einer Netto-Arbeitszeit, die wesentlich geringer ist,
als Sie angenommen hatten. Das macht aber nichts -- denn dies ist dafür
nun eine realistische Annahme. Bedenken Sie auch, dass anspruchsvolle
Arbeiten wie Lesen und Schreiben kaum „zwischendurch`` in einer
freigebliebenen Stunde möglich sind, sondern dass Sie Blöcke von
mehreren Stunden brauchen, um sich voll auf Ihre Arbeit konzentrieren zu
können (was nicht heißt, dass Sie dabei nicht kurze Entspannungspausen
einlegen sollen).

Das folgende Beispiel für einen Zeitplan orientiert sich an einer
Magister- oder Diplomarbeit, für die ca. sechs Monate zur Verfügung
stehen. Als wöchentliche Netto- Arbeitszeit wurden durchschnittlich 40
Stunden angenommen (also relativ viel!), bei fünf Arbeitstagen pro
Woche:

\subsubsection{Zeitplan für eine größere Arbeit (z. B.
Diplomarbeit)}\label{zeitplan-fur-eine-groere-arbeit-z.-b.-diplomarbeit}

\begin{longtable}[]{@{}lll@{}}
\caption{\textbf{Aktionstabelle: Zeitplan für eine größere Arbeit (z. B.
Diplomarbeit)}}\tabularnewline
\toprule
\begin{minipage}[b]{0.16\columnwidth}\raggedright\strut
Sie rechnen~\ldots{}\strut
\end{minipage} & \begin{minipage}[b]{0.35\columnwidth}\raggedright\strut
für \ldots{}\strut
\end{minipage} & \begin{minipage}[b]{0.41\columnwidth}\raggedright\strut
vorausgesetzt, dass Sie \ldots{}\strut
\end{minipage}\tabularnewline
\midrule
\endfirsthead
\toprule
\begin{minipage}[b]{0.16\columnwidth}\raggedright\strut
Sie rechnen~\ldots{}\strut
\end{minipage} & \begin{minipage}[b]{0.35\columnwidth}\raggedright\strut
für \ldots{}\strut
\end{minipage} & \begin{minipage}[b]{0.41\columnwidth}\raggedright\strut
vorausgesetzt, dass Sie \ldots{}\strut
\end{minipage}\tabularnewline
\midrule
\endhead
\begin{minipage}[t]{0.16\columnwidth}\raggedright\strut
zwei Wochen\strut
\end{minipage} & \begin{minipage}[t]{0.35\columnwidth}\raggedright\strut
erste Materialsammlung\strut
\end{minipage} & \begin{minipage}[t]{0.41\columnwidth}\raggedright\strut
bereits eine Idee haben\strut
\end{minipage}\tabularnewline
\begin{minipage}[t]{0.16\columnwidth}\raggedright\strut
eine Woche\strut
\end{minipage} & \begin{minipage}[t]{0.35\columnwidth}\raggedright\strut
Konzepterstellung\strut
\end{minipage} & \begin{minipage}[t]{0.41\columnwidth}\raggedright\strut
danach ein Gespräch mit dem/der Betreuer/in führen\strut
\end{minipage}\tabularnewline
\begin{minipage}[t]{0.16\columnwidth}\raggedright\strut
zwei Wochen\strut
\end{minipage} & \begin{minipage}[t]{0.35\columnwidth}\raggedright\strut
Konzeptüberarbeitung\strut
\end{minipage} & \begin{minipage}[t]{0.41\columnwidth}\raggedright\strut
das Konzept nach dem Betreuungsgespräch ändern müssen\strut
\end{minipage}\tabularnewline
\begin{minipage}[t]{0.16\columnwidth}\raggedright\strut
drei Wochen\strut
\end{minipage} & \begin{minipage}[t]{0.35\columnwidth}\raggedright\strut
Literatursuche, -auswahl und -beschaffung\strut
\end{minipage} & \begin{minipage}[t]{0.41\columnwidth}\raggedright\strut
„grünes Licht`` bekommen haben und ca. 100 Seiten pro Tag überfliegen
können\strut
\end{minipage}\tabularnewline
\begin{minipage}[t]{0.16\columnwidth}\raggedright\strut
neun Wochen\strut
\end{minipage} & \begin{minipage}[t]{0.35\columnwidth}\raggedright\strut
Lesen und notieren\strut
\end{minipage} & \begin{minipage}[t]{0.41\columnwidth}\raggedright\strut
ca. 50 Seiten pro Tag analytischlesen können, inkl. Notizen machen\strut
\end{minipage}\tabularnewline
\begin{minipage}[t]{0.16\columnwidth}\raggedright\strut
vier Wochen\strut
\end{minipage} & \begin{minipage}[t]{0.35\columnwidth}\raggedright\strut
Rohfassung schreiben\strut
\end{minipage} & \begin{minipage}[t]{0.41\columnwidth}\raggedright\strut
bei 80-100 Seiten Gesamtumfang ca. 4-5 Seiten pro Tag schreiben\strut
\end{minipage}\tabularnewline
\begin{minipage}[t]{0.16\columnwidth}\raggedright\strut
drei Wochen\strut
\end{minipage} & \begin{minipage}[t]{0.35\columnwidth}\raggedright\strut
Überarbeitung der Rohfassung\strut
\end{minipage} & \begin{minipage}[t]{0.41\columnwidth}\raggedright\strut
Einleitung, Schluss, Übergänge noch formulieren müssen\strut
\end{minipage}\tabularnewline
\begin{minipage}[t]{0.16\columnwidth}\raggedright\strut
eine Woche\strut
\end{minipage} & \begin{minipage}[t]{0.35\columnwidth}\raggedright\strut
Fertigstellung\strut
\end{minipage} & \begin{minipage}[t]{0.41\columnwidth}\raggedright\strut
alle inhaltlichen Teile vollständig haben; alle bibliografischen Angaben
verfügbar haben\strut
\end{minipage}\tabularnewline
\begin{minipage}[t]{0.16\columnwidth}\raggedright\strut
\textbf{25 Wochen}\strut
\end{minipage} & \begin{minipage}[t]{0.35\columnwidth}\raggedright\strut
\textbf{die gesamte Arbeit bis zur Abgabe}\strut
\end{minipage} & \begin{minipage}[t]{0.41\columnwidth}\raggedright\strut
\strut
\end{minipage}\tabularnewline
\bottomrule
\end{longtable}

\subsubsection{Zeitplan für eine
Semesterarbeit}\label{zeitplan-fur-eine-semesterarbeit}

Zum Vergleich dazu zeigen wir hier noch einen möglichen Zeitplan für
eine kleinere Arbeit (z. B. Semesterarbeit), die nebenher geschrieben
werden muss. Hier wurden als wöchentliche Arbeitszeit ca. 10 Stunden
angenommen, und als Zeitbudget zehn Wochen.

\begin{longtable}[]{@{}lll@{}}
\caption{\textbf{Aktionstabelle: Zeitplan für eine kleinere Arbeit (z.
B. Semesterarbeit)}}\tabularnewline
\toprule
\begin{minipage}[b]{0.14\columnwidth}\raggedright\strut
Sie rechnen~\ldots{}\strut
\end{minipage} & \begin{minipage}[b]{0.40\columnwidth}\raggedright\strut
für \ldots{}\strut
\end{minipage} & \begin{minipage}[b]{0.37\columnwidth}\raggedright\strut
Sie sollten \ldots{}\strut
\end{minipage}\tabularnewline
\midrule
\endfirsthead
\toprule
\begin{minipage}[b]{0.14\columnwidth}\raggedright\strut
Sie rechnen~\ldots{}\strut
\end{minipage} & \begin{minipage}[b]{0.40\columnwidth}\raggedright\strut
für \ldots{}\strut
\end{minipage} & \begin{minipage}[b]{0.37\columnwidth}\raggedright\strut
Sie sollten \ldots{}\strut
\end{minipage}\tabularnewline
\midrule
\endhead
\begin{minipage}[t]{0.14\columnwidth}\raggedright\strut
zwei Wochen\strut
\end{minipage} & \begin{minipage}[t]{0.40\columnwidth}\raggedright\strut
erste Materialsammlung und Konzipieren der Arbeit\strut
\end{minipage} & \begin{minipage}[t]{0.37\columnwidth}\raggedright\strut
frühzeitig das Thema und damit die Literatursuche eingrenzen\strut
\end{minipage}\tabularnewline
\begin{minipage}[t]{0.14\columnwidth}\raggedright\strut
eine Woche\strut
\end{minipage} & \begin{minipage}[t]{0.40\columnwidth}\raggedright\strut
ergänzende systematische Literatursuche und -beschaffung\strut
\end{minipage} & \begin{minipage}[t]{0.37\columnwidth}\raggedright\strut
sich auf rasch zugängliche Literatur beschränken und die Suche streng
zeitlich begrenzen\strut
\end{minipage}\tabularnewline
\begin{minipage}[t]{0.14\columnwidth}\raggedright\strut
drei Wochen\strut
\end{minipage} & \begin{minipage}[t]{0.40\columnwidth}\raggedright\strut
Lesen und Notieren\strut
\end{minipage} & \begin{minipage}[t]{0.37\columnwidth}\raggedright\strut
Notizen möglichst so verfassen, dass sie im Text verwendet werden
können\strut
\end{minipage}\tabularnewline
\begin{minipage}[t]{0.14\columnwidth}\raggedright\strut
drei Wochen\strut
\end{minipage} & \begin{minipage}[t]{0.40\columnwidth}\raggedright\strut
Rohfassung schreiben\strut
\end{minipage} & \begin{minipage}[t]{0.37\columnwidth}\raggedright\strut
bei 30 Seiten Gesamtumfang durchschnittlich 2 Seiten pro Tag
schreiben\strut
\end{minipage}\tabularnewline
\begin{minipage}[t]{0.14\columnwidth}\raggedright\strut
eine Woche\strut
\end{minipage} & \begin{minipage}[t]{0.40\columnwidth}\raggedright\strut
Überarbeitung und Fertigstellung\strut
\end{minipage} & \begin{minipage}[t]{0.37\columnwidth}\raggedright\strut
Formvorschriften, Zitierweise usw. schon beim Schreiben
berücksichtigen\strut
\end{minipage}\tabularnewline
\begin{minipage}[t]{0.14\columnwidth}\raggedright\strut
\textbf{zehn Wochen}\strut
\end{minipage} & \begin{minipage}[t]{0.40\columnwidth}\raggedright\strut
\textbf{die gesamte Arbeit bis zur Abgabe}\strut
\end{minipage} & \begin{minipage}[t]{0.37\columnwidth}\raggedright\strut
\strut
\end{minipage}\tabularnewline
\bottomrule
\end{longtable}

\begin{longtable}[]{@{}lll@{}}
\caption{\label{tab:unnamed-chunk-4}Möglicher Zeitplan für eine
Semesterarbeit}\tabularnewline
\toprule
Sie rechnen\ldots{} & für\ldots{} & Sie sollten\ldots{}\tabularnewline
\midrule
\endfirsthead
\toprule
Sie rechnen\ldots{} & für\ldots{} & Sie sollten\ldots{}\tabularnewline
\midrule
\endhead
zwei Wochen & erste Materialsammlung und Konzipieren der Arbeit &
frühzeitig das Thema und damit die Literatursuche
eingrenzen\tabularnewline
eine Woche & ergänzende systematische Literatursuche und -beschaffung &
sich auf rasch zugängliche Literatur beschränken und die Suche streng
zeitlich begrenzen\tabularnewline
drei Wochen & Lesen und Notieren & Notizen möglichst so verfassen, dass
sie im Text verwendet werden können\tabularnewline
drei Wochen & Rohfassung schreiben & bei 30 Seiten Gesamtumfang
durchschnittlich 2 Seiten pro Tag schreiben\tabularnewline
eine Woche & Überarbeitung und Fertigstellung & Formvorschriften,
Zitierweise usw. schon beim Schreiben berücksichtigen\tabularnewline
\textbf{zehn Wochen} & \textbf{die gesamte Arbeit bis zur Abgabe}
&\tabularnewline
\bottomrule
\end{longtable}

\subsubsection{Rahmenbedingungen
abklären}\label{rahmenbedingungen-abklaren}

\begin{longtable}[]{@{}ll@{}}
\caption{\label{tab:unnamed-chunk-5}Rahmenbedingung, die es zu klären
gibt}\tabularnewline
\toprule
& Wieviel Zeit steht mir für die Arbeit zur Verfügung? (Abgabetermin,
Bearbeitungsfrist, eigene Planung)\tabularnewline
& Wieviel Zeit werde ich für die einzelnen Phasen benötigen, bzw.
wieviel darf ich mir erlauben?\tabularnewline
& Wieviel Geld steht mir zur Verfügung z. B. für Anschaffung von
Literatur, Kopien, Reisen, Software, Internet-Zugang
usw.?\tabularnewline
& Welche Infrastruktur steht zur Verfügung, mit welchen Beschränkungen?
(z. B. PC-Benutzerräume -- Ausstattung, Auslastung? Bibliotheken --
Öffnungszeiten, Urlaubssperren, Entleihfristen?)\tabularnewline
& Wie ist die Betreuung organisiert? (regelmäßige Treffen oder
Sprechstunden, Terminvereinbarung, Abwesenheit \ldots{})\tabularnewline
\bottomrule
\end{longtable}

Der Einfluss dieser äußeren Bedingungen auf die Themenwahl, aber auch
auf Methoden, Ansprüche, Umfang usw. kann sich in verschiedenen
Entscheidungen niederschlagen. Die nachfolgende Tabelle listet einige --
längst nicht alle -- Schwierigkeiten auf:

Aktionstabelle: Auswirkungen der Rahmenbedingungen Sie haben z. B.
\ldots{} Sie entscheiden: \ldots{} und handeln (Beispiele): eine fixe
Abgabefrist für die Arbeit Ist das Thema in diesem Zeitrahmen machbar?
Thema eingrenzen, Methode ändern (z. B. nur Literaturstudie oder nur
empirische Arbeit) beschränkten Internet- Zugang Wann und wofür
verwenden Sie das Internet, wie lässt sich die beschränkte Zeit sinnvoll
nutzen? Zeit des Internet-Zugang planen (Uni, Tageszeit) Hilfsmittel
kaufen (z. B. Offline Browser) beschränkten Zugang zu PCs, Druckern etc.
Wann und wie wird dieser Zugang am effektivsten genutzt? Lässt sich der
Zustand ändern? eigenen PC kaufen, Arbeitsplatz reservieren, PC leihen
Literatur, die nur über Fernleihe verfügbar ist Ist das Thema durch
diese Zeitverzögerung im vorgesehenen Zeitrahmen machbar? Literatur
sofort anfordern, anderes Thema wählen Literatur, die nicht entliehen
werden darf Öffnungszeiten ausreichend? Literatur kaufen, kopieren die
Erfahrung, dass Sie für eine bestimmte Tätigkeit viel Zeit brauchen,
etwa zum Schreiben Wie viel Zeit dürfen (innerhalb eines festen
Zeitrahmens) die anderen Tätigkeiten maximal beanspruchen? Arbeitsphase
entsprechend einplanen, Hilfe organisieren (z. B. beim Korrekturlesen)
einen Betreuer, der schwer erreichbar ist Gibt es Abhilfen oder
Alternativen? Sie vereinbaren Beratungsmodalitäten (Termine, e- Mail
etc.), Sie suchen sich eine andere Betreuerin

Manchmal können sich während einer (längeren) Arbeitsdauer die
Rahmenbedingungen auch ändern. Die Handlungsentscheidungen dieser
Tabelle sind daher nicht nur so früh wie möglich zu treffen, sondern an
Hand der aktuellen Rahmenbedingungen auch immer wieder zu überprüfen.

\subsection{Eine Fragestellung
erarbeiten}\label{eine-fragestellung-erarbeiten}

Wir betrachten hier den Fall, dass man sich das Thema für eine
wissenschaftliche Arbeit selbst suchen kann oder muss. Meist sind auch
die von der Betreuerin vorgeschlagenen Themen erst grob gefasst, sodass
genügend Spielraum für eigene Gestaltung -- und das Einbringen eigener
Interessen -- bleibt. Auch in diesen Fällen sollte man also ähnlich wie
nachfolgend beschrieben vorgehen.

Nachdem eine Idee aufgetaucht ist, besteht der erste Schritt darin, sie
in allen Facetten zu erkunden, um möglichst vielfältige Fragen zu
generieren. In dieser Phase geht es darum, ein geeignetes Problem zu
finden, indem die Idee systematisch exploriert und die Fragestellung
„geöffnet`` wird.

\subsubsection{Thematik erschließen,
„öffnen``}\label{thematik-erschlieen-offnen}

Meistens weiß man viel mehr über ein Thema oder ein Sachgebiet, als man
selbst glaubt. Um dieses Wissen zu aktivieren, sind Methoden
empfehlenswert, die extra dafür entwickelt wurden, um das assoziative
und kreative Denken zu fördern. Mit diesen Techniken lässt sich eine
erste Idee nach vielen Richtungen hin genauer entwickeln („öffnen``):

Beim Brainstorming schreibt man alle Begriffe ungeordnet auf, die einem
zum Thema einfallen. Eine wichtige Regel dabei ist, dass man sich
während dieses Prozesses jedes Urteil und jede Analyse verbietet und
alles hinschreibt, was einem durch den Kopf geht. Erst später, wenn
einem nichts mehr Neues einfällt, wird die so entstandene Liste
geordnet, strukturiert und weiter bearbeitet.

Brainstorming ist eine Methode, die vor allem zur Generierung von Ideen
in Gruppen verwendet wird. In diesem Fall werden die Ideen von den
Teilnehmerinnen laut ausgesprochen und von einer Person mitgeschrieben.
Mehrere Köpfe sind besser als einer -- bekanntlich sogar besser als die
Summe der einzelnen Leistungen. Brainstorming allein zu betreiben, ist
daher nur eine Notlösung. Wenn es irgendwie möglich ist, organisieren
Sie ein Gruppen-Brainstorming zu Ihrem Thema. Das kann auch im
informellen Rahmen mit ein paar Freundinnen, z. B. im Kaffeehaus oder in
der Kneipe, geschehen: je lockerer die Atmosphäre, desto freier sind die
Assoziationen. Auch witzige oder auf den ersten Blick unsinnige
Vorschläge haben gleiches Recht -- manchmal sind gerade sie besonders
anregend.

Beim Mind-Mapping werden die Gedanken dagegen nicht in einer Liste
aufgeschrieben, sondern in einer frei gezeichneten, oft baum- oder
sternartigen Struktur festgehalten. Durch das Zeichnen wird gleich auch
eine Struktur entwickelt, außerdem bleibt man dabei nicht auf das verbal
ausdrückbare Wissen beschränkt. Das spielerische Element wird dabei
betont.

Man braucht dazu ein Blatt Papier und bunte Stifte (oder
Mind-Mapping-Software). In die Mitte oder unten wird das Stichwort oder
Thema geschrieben, als Zentrum bzw. Wurzel. Alle Aspekte, Stichworte,
Fragen usw., die einem dazu einfallen, werden nun dazugeschrieben und
-gezeichnet. Das können, müssen aber nicht, die „Äste`` des Baumes oder
Strahlen des Sterns werden. Diese Aspekte können weitere Fragen und
Assoziationen aufwerfen, sodass eine hierarchische Struktur mit Zweigen
an den Ästen entstehen kann. Umgekehrt kann es aber auch vorkommen, dass
zuerst verschiedene Begriffe einzeln dastehen und man versucht, sie
spielerisch-zeichnend in Verbindung zueinander zu bringen.

Oft lohnt es sich auch, für sich selbst zu ergründen, wie man auf die
Idee gestoßen ist, warum sie interessant erscheint, was man für Ängste
oder Wünsche damit verbindet. Fangen Sie einfach an, über Ihr Thema alle
Gedanken aufzuschreiben. Kümmern Sie sich vorerst weder um Schreib- oder
Tippfehler, noch um den Stil; sondern lassen Sie Ihren Gedanken freien
Lauf. Es kann dabei die Fragestellung auftauchen, die eigentlich zu
diesem Thema motiviert.

Ein Projekttagebuch, das auch ein schlichtes Notizheft sein kann, dient
dazu, über längere Zeiträume hinweg immer wieder Fragen, Ideen,
Bemerkungen usw. zum Thema aufzuschreiben. Mit dieser Methode werden
sich vor allem diejenigen anfreunden, die auch sonst ihre Gedanken gerne
schreibend generieren und ordnen.

\subsubsection{Material sammeln}\label{material-sammeln}

Nach der Phase der Ideenfindung wird eine erste Recherche durchgeführt
(siehe Kapitel „Recherchieren``). Bei dieser vorläufigen
Materialsammlung geht es weder um Vollständigkeit noch um Genauigkeit,
sondern bloß darum, einen ersten, möglichst breiten Überblick zu
bekommen, wo und wie etwas zum Thema gesagt und veröffentlicht wurde.
Der Zweck dieser Rundschau ist es, sich stichprobenartig mit dem Thema
vertraut zu machen und es von verschiedenen Seiten zu betrachten, um
später den eigenen Zugang dazu abgrenzen und begründen zu können.

Nutzen Sie alle Sinne, Möglichkeiten und Quellen, um zu Informationen
über Ihren Gegenstand zu gelangen. Das beginnt beim Einholen von Tipps
und Meinungen von erfahreneren Personen oder der Betreuerin, geht über
die „klassischen`` Hilfsmittel des wissenschaftlichen Arbeitens (Lexika,
Lehr- und Fachbücher) und die „traditionelle`` Literaturrecherche bis
hin zu den neuen Möglichkeiten, die das Internet bietet.

Achten Sie beim Sammeln des Materials nicht nur auf die Inhalte, sondern
auch auf die Akteure (Personen, Institutionen,\ldots{}), die diese
Positionen vertreten. Das ist besonders für aktuelle Bezüge
(Kontaktnahme, Interviews,\ldots{}) und die Orientierung im
wissenschaftlichen Diskurs (Meinungsführer, Kontrahenten, Schulen,
\ldots{}) wichtig.

Mit der globalen Vernetzung hat sich heute eine völlig neue Dimension
für Studierende und Wissenschafter geöffnet: Relativ einfach, mit wenig
Kosten und geringem Zeitaufwand können Sie sich weltweit z. B. über
aktuelle Projekte informieren (Recherche im Internet, Besuch der
Homepage des jeweiligen Wissenschaftlers, Instituts etc.) oder eine
konkrete Anfrage starten (siehe Kapitel „Recherchieren im Internet``).

\begin{longtable}[]{@{}lc@{}}
\caption{\textbf{Prüfliste ``Social Inquiry 1''}}\tabularnewline
\toprule
\begin{minipage}[b]{0.66\columnwidth}\raggedright\strut
Fragen, die geprüft werden sollten\strut
\end{minipage} & \begin{minipage}[b]{0.28\columnwidth}\centering\strut
Ja?\strut
\end{minipage}\tabularnewline
\midrule
\endfirsthead
\toprule
\begin{minipage}[b]{0.66\columnwidth}\raggedright\strut
Fragen, die geprüft werden sollten\strut
\end{minipage} & \begin{minipage}[b]{0.28\columnwidth}\centering\strut
Ja?\strut
\end{minipage}\tabularnewline
\midrule
\endhead
\begin{minipage}[t]{0.66\columnwidth}\raggedright\strut
Wen kann ich aus meinem Bekanntenkreis zur Thematik fragen? (Betreuerin,
Kollegin, fachfremde Personen,\ldots{})\strut
\end{minipage} & \begin{minipage}[t]{0.28\columnwidth}\centering\strut
\strut
\end{minipage}\tabularnewline
\begin{minipage}[t]{0.66\columnwidth}\raggedright\strut
Wen könnte ich außerhalb meines Bekanntenkreises fragen, anschreiben,
interviewen?\strut
\end{minipage} & \begin{minipage}[t]{0.28\columnwidth}\centering\strut
\strut
\end{minipage}\tabularnewline
\begin{minipage}[t]{0.66\columnwidth}\raggedright\strut
Welche Personen haben sich zu meiner Problematik bereits
profiliert?\strut
\end{minipage} & \begin{minipage}[t]{0.28\columnwidth}\centering\strut
\strut
\end{minipage}\tabularnewline
\begin{minipage}[t]{0.66\columnwidth}\raggedright\strut
Wer vertritt (im wissenschaftlichen Diskurs, in einer aktuellen
Kontroverse) welche Position?\strut
\end{minipage} & \begin{minipage}[t]{0.28\columnwidth}\centering\strut
\strut
\end{minipage}\tabularnewline
\begin{minipage}[t]{0.66\columnwidth}\raggedright\strut
Gibt es eine Meinungsführerin?\strut
\end{minipage} & \begin{minipage}[t]{0.28\columnwidth}\centering\strut
\strut
\end{minipage}\tabularnewline
\begin{minipage}[t]{0.66\columnwidth}\raggedright\strut
Welche Arbeits- und Forschungsgruppen arbeiten an meinem Thema?\strut
\end{minipage} & \begin{minipage}[t]{0.28\columnwidth}\centering\strut
\strut
\end{minipage}\tabularnewline
\begin{minipage}[t]{0.66\columnwidth}\raggedright\strut
Welche Institution (Institut, Labor, Forschungseinrichtung) könnte ich
besuchen (evt. virtuell über die Homepage im Internet)?\strut
\end{minipage} & \begin{minipage}[t]{0.28\columnwidth}\centering\strut
\strut
\end{minipage}\tabularnewline
\begin{minipage}[t]{0.66\columnwidth}\raggedright\strut
Wie könnte ich kompetente Personen motivieren, für mich ihre kostbare
Zeit zu opfern?\strut
\end{minipage} & \begin{minipage}[t]{0.28\columnwidth}\centering\strut
\strut
\end{minipage}\tabularnewline
\begin{minipage}[t]{0.66\columnwidth}\raggedright\strut
Gibt es dazu eine gute (passende) Gelegenheit (Konferenz, Vortrag an
meiner Uni etc.)?\strut
\end{minipage} & \begin{minipage}[t]{0.28\columnwidth}\centering\strut
\strut
\end{minipage}\tabularnewline
\bottomrule
\end{longtable}

Auch wenn es heute (technisch) relativ einfach ist, mit den Akteurinnen
der Wissenschaft in Kontakt zu treten, so bedeutet das nicht
automatisch, dass Sie auch eine Antwort auf Ihre Frage(n) bekommen:
Namhafte Leute (Wissenschafterinnen, Journalistinnen) sind viel
beschäftigt und Sie müssen sich schon einen guten Grund einfallen
lassen, wenn Ihre Anfrage nicht sang- und klanglos in den Weiten des
Cyberspace verschwinden soll (siehe Kapitel „Kooperieren``).

\begin{longtable}[]{@{}lr@{}}
\caption{\textbf{Prüfliste ``Social Inquiry 2''}}\tabularnewline
\toprule
\begin{minipage}[b]{0.66\columnwidth}\raggedright\strut
Fragen, die geprüft werden sollten\strut
\end{minipage} & \begin{minipage}[b]{0.28\columnwidth}\raggedleft\strut
Ja?\strut
\end{minipage}\tabularnewline
\midrule
\endfirsthead
\toprule
\begin{minipage}[b]{0.66\columnwidth}\raggedright\strut
Fragen, die geprüft werden sollten\strut
\end{minipage} & \begin{minipage}[b]{0.28\columnwidth}\raggedleft\strut
Ja?\strut
\end{minipage}\tabularnewline
\midrule
\endhead
\begin{minipage}[t]{0.66\columnwidth}\raggedright\strut
Wen kann ich aus meinem Bekanntenkreis zur Thematik fragen? (Betreuerin,
Kollegin, fachfremde Personen,\ldots{})\strut
\end{minipage} & \begin{minipage}[t]{0.28\columnwidth}\raggedleft\strut
\strut
\end{minipage}\tabularnewline
\begin{minipage}[t]{0.66\columnwidth}\raggedright\strut
Wen könnte ich außerhalb meines Bekanntenkreises fragen, anschreiben,
interviewen?\strut
\end{minipage} & \begin{minipage}[t]{0.28\columnwidth}\raggedleft\strut
\strut
\end{minipage}\tabularnewline
\begin{minipage}[t]{0.66\columnwidth}\raggedright\strut
Welche Personen haben sich zu meiner Problematik bereits
profiliert?\strut
\end{minipage} & \begin{minipage}[t]{0.28\columnwidth}\raggedleft\strut
\strut
\end{minipage}\tabularnewline
\begin{minipage}[t]{0.66\columnwidth}\raggedright\strut
Wer vertritt (im wissenschaftlichen Diskurs, in einer aktuellen
Kontroverse) welche Position?\strut
\end{minipage} & \begin{minipage}[t]{0.28\columnwidth}\raggedleft\strut
\strut
\end{minipage}\tabularnewline
\begin{minipage}[t]{0.66\columnwidth}\raggedright\strut
Gibt es eine Meinungsführerin?\strut
\end{minipage} & \begin{minipage}[t]{0.28\columnwidth}\raggedleft\strut
\strut
\end{minipage}\tabularnewline
\begin{minipage}[t]{0.66\columnwidth}\raggedright\strut
Welche Arbeits- und Forschungsgruppen arbeiten an meinem Thema?\strut
\end{minipage} & \begin{minipage}[t]{0.28\columnwidth}\raggedleft\strut
\strut
\end{minipage}\tabularnewline
\begin{minipage}[t]{0.66\columnwidth}\raggedright\strut
Welche Institution (Institut, Labor, Forschungseinrichtung) könnte ich
besuchen (evt. virtuell über die Homepage im Internet)?\strut
\end{minipage} & \begin{minipage}[t]{0.28\columnwidth}\raggedleft\strut
\strut
\end{minipage}\tabularnewline
\begin{minipage}[t]{0.66\columnwidth}\raggedright\strut
Wie könnte ich kompetente Personen motivieren, für mich ihre kostbare
Zeit zu opfern?\strut
\end{minipage} & \begin{minipage}[t]{0.28\columnwidth}\raggedleft\strut
\strut
\end{minipage}\tabularnewline
\begin{minipage}[t]{0.66\columnwidth}\raggedright\strut
Gibt es dazu eine gute (passende) Gelegenheit (Konferenz, Vortrag an
meiner Uni etc.)?\strut
\end{minipage} & \begin{minipage}[t]{0.28\columnwidth}\raggedleft\strut
\strut
\end{minipage}\tabularnewline
\bottomrule
\end{longtable}

Überlegen Sie sich daher, wie Sie Ihr (virtuelles) Gegenüber zu einer
Antwort motivieren können. Am Besten ist natürlich, wenn Sie nicht nur
etwas wollen, sondern auch etwas anzubieten haben wie z. B. Übermittlung
der Arbeitsergebnisse, an denen die Expertin Interesse haben könnte.
Ihre Chancen auf eine Antwort steigen auch, wenn Sie den Kontakt durch
eine andere Person, die die Angefragte kennt bzw. ihr persönlich
verbunden ist, aufnehmen können.

\begin{longtable}[]{@{}ll@{}}
\caption{\textbf{Prüfliste ``Social Inquiry 3''}}\tabularnewline
\toprule
\begin{minipage}[b]{0.27\columnwidth}\raggedright\strut
Ja?\strut
\end{minipage} & \begin{minipage}[b]{0.67\columnwidth}\raggedright\strut
Fragen, die geprüft werden sollten\strut
\end{minipage}\tabularnewline
\midrule
\endfirsthead
\toprule
\begin{minipage}[b]{0.27\columnwidth}\raggedright\strut
Ja?\strut
\end{minipage} & \begin{minipage}[b]{0.67\columnwidth}\raggedright\strut
Fragen, die geprüft werden sollten\strut
\end{minipage}\tabularnewline
\midrule
\endhead
\begin{minipage}[t]{0.27\columnwidth}\raggedright\strut
\strut
\end{minipage} & \begin{minipage}[t]{0.67\columnwidth}\raggedright\strut
Wen kann ich aus meinem Bekanntenkreis zur Thematik fragen? (Betreuerin,
Kollegin, fachfremde Personen,\ldots{})\strut
\end{minipage}\tabularnewline
\begin{minipage}[t]{0.27\columnwidth}\raggedright\strut
\strut
\end{minipage} & \begin{minipage}[t]{0.67\columnwidth}\raggedright\strut
Wen könnte ich außerhalb meines Bekanntenkreises fragen, anschreiben,
interviewen?\strut
\end{minipage}\tabularnewline
\begin{minipage}[t]{0.27\columnwidth}\raggedright\strut
\strut
\end{minipage} & \begin{minipage}[t]{0.67\columnwidth}\raggedright\strut
Welche Personen haben sich zu meiner Problematik bereits
profiliert?\strut
\end{minipage}\tabularnewline
\begin{minipage}[t]{0.27\columnwidth}\raggedright\strut
\strut
\end{minipage} & \begin{minipage}[t]{0.67\columnwidth}\raggedright\strut
Wer vertritt (im wissenschaftlichen Diskurs, in einer aktuellen
Kontroverse) welche Position?\strut
\end{minipage}\tabularnewline
\begin{minipage}[t]{0.27\columnwidth}\raggedright\strut
\strut
\end{minipage} & \begin{minipage}[t]{0.67\columnwidth}\raggedright\strut
Gibt es eine Meinungsführerin?\strut
\end{minipage}\tabularnewline
\begin{minipage}[t]{0.27\columnwidth}\raggedright\strut
\strut
\end{minipage} & \begin{minipage}[t]{0.67\columnwidth}\raggedright\strut
Welche Arbeits- und Forschungsgruppen arbeiten an meinem Thema?\strut
\end{minipage}\tabularnewline
\begin{minipage}[t]{0.27\columnwidth}\raggedright\strut
\strut
\end{minipage} & \begin{minipage}[t]{0.67\columnwidth}\raggedright\strut
Welche Institution (Institut, Labor, Forschungseinrichtung) könnte ich
besuchen (evt. virtuell über die Homepage im Internet)?\strut
\end{minipage}\tabularnewline
\begin{minipage}[t]{0.27\columnwidth}\raggedright\strut
\strut
\end{minipage} & \begin{minipage}[t]{0.67\columnwidth}\raggedright\strut
Wie könnte ich kompetente Personen motivieren, für mich ihre kostbare
Zeit zu opfern?\strut
\end{minipage}\tabularnewline
\begin{minipage}[t]{0.27\columnwidth}\raggedright\strut
\strut
\end{minipage} & \begin{minipage}[t]{0.67\columnwidth}\raggedright\strut
Gibt es dazu eine gute (passende) Gelegenheit (Konferenz, Vortrag an
meiner Uni etc.)?\strut
\end{minipage}\tabularnewline
\bottomrule
\end{longtable}

Überlegen Sie sich schon bei dieser ersten Recherche ein System, mit dem
Sie gefundene Literatur, E-Mail- und Internet-Adressen usw. so ordnen
und aufbewahren, dass Sie später bei der systematischen Materialsammlung
darauf zurückgreifen können. (siehe „Material ordnen``)

\begin{longtable}[]{@{}cl@{}}
\caption{\textbf{Prüfliste ``Social Inquiry 4''}}\tabularnewline
\toprule
\begin{minipage}[b]{0.27\columnwidth}\centering\strut
Ja?\strut
\end{minipage} & \begin{minipage}[b]{0.67\columnwidth}\raggedright\strut
Fragen, die geprüft werden sollten\strut
\end{minipage}\tabularnewline
\midrule
\endfirsthead
\toprule
\begin{minipage}[b]{0.27\columnwidth}\centering\strut
Ja?\strut
\end{minipage} & \begin{minipage}[b]{0.67\columnwidth}\raggedright\strut
Fragen, die geprüft werden sollten\strut
\end{minipage}\tabularnewline
\midrule
\endhead
\begin{minipage}[t]{0.27\columnwidth}\centering\strut
\strut
\end{minipage} & \begin{minipage}[t]{0.67\columnwidth}\raggedright\strut
Wen kann ich aus meinem Bekanntenkreis zur Thematik fragen? (Betreuerin,
Kollegin, fachfremde Personen,\ldots{})\strut
\end{minipage}\tabularnewline
\begin{minipage}[t]{0.27\columnwidth}\centering\strut
\strut
\end{minipage} & \begin{minipage}[t]{0.67\columnwidth}\raggedright\strut
Wen könnte ich außerhalb meines Bekanntenkreises fragen, anschreiben,
interviewen?\strut
\end{minipage}\tabularnewline
\begin{minipage}[t]{0.27\columnwidth}\centering\strut
\strut
\end{minipage} & \begin{minipage}[t]{0.67\columnwidth}\raggedright\strut
Welche Personen haben sich zu meiner Problematik bereits
profiliert?\strut
\end{minipage}\tabularnewline
\begin{minipage}[t]{0.27\columnwidth}\centering\strut
\strut
\end{minipage} & \begin{minipage}[t]{0.67\columnwidth}\raggedright\strut
Wer vertritt (im wissenschaftlichen Diskurs, in einer aktuellen
Kontroverse) welche Position?\strut
\end{minipage}\tabularnewline
\begin{minipage}[t]{0.27\columnwidth}\centering\strut
\strut
\end{minipage} & \begin{minipage}[t]{0.67\columnwidth}\raggedright\strut
Gibt es eine Meinungsführerin?\strut
\end{minipage}\tabularnewline
\begin{minipage}[t]{0.27\columnwidth}\centering\strut
\strut
\end{minipage} & \begin{minipage}[t]{0.67\columnwidth}\raggedright\strut
Welche Arbeits- und Forschungsgruppen arbeiten an meinem Thema?\strut
\end{minipage}\tabularnewline
\begin{minipage}[t]{0.27\columnwidth}\centering\strut
\strut
\end{minipage} & \begin{minipage}[t]{0.67\columnwidth}\raggedright\strut
Welche Institution (Institut, Labor, Forschungseinrichtung) könnte ich
besuchen (evt. virtuell über die Homepage im Internet)?\strut
\end{minipage}\tabularnewline
\begin{minipage}[t]{0.27\columnwidth}\centering\strut
\strut
\end{minipage} & \begin{minipage}[t]{0.67\columnwidth}\raggedright\strut
Wie könnte ich kompetente Personen motivieren, für mich ihre kostbare
Zeit zu opfern?\strut
\end{minipage}\tabularnewline
\begin{minipage}[t]{0.27\columnwidth}\centering\strut
\strut
\end{minipage} & \begin{minipage}[t]{0.67\columnwidth}\raggedright\strut
Gibt es dazu eine gute (passende) Gelegenheit (Konferenz, Vortrag an
meiner Uni etc.)?\strut
\end{minipage}\tabularnewline
\bottomrule
\end{longtable}

\subsubsection{Material sichten und
bewerten}\label{material-sichten-und-bewerten}

\begin{longtable}[]{@{}rl@{}}
\caption{\textbf{Prüfliste ``Social Inquiry 5''}}\tabularnewline
\toprule
\begin{minipage}[b]{0.27\columnwidth}\raggedleft\strut
Ja?\strut
\end{minipage} & \begin{minipage}[b]{0.67\columnwidth}\raggedright\strut
Fragen, die geprüft werden sollten\strut
\end{minipage}\tabularnewline
\midrule
\endfirsthead
\toprule
\begin{minipage}[b]{0.27\columnwidth}\raggedleft\strut
Ja?\strut
\end{minipage} & \begin{minipage}[b]{0.67\columnwidth}\raggedright\strut
Fragen, die geprüft werden sollten\strut
\end{minipage}\tabularnewline
\midrule
\endhead
\begin{minipage}[t]{0.27\columnwidth}\raggedleft\strut
\strut
\end{minipage} & \begin{minipage}[t]{0.67\columnwidth}\raggedright\strut
Wen kann ich aus meinem Bekanntenkreis zur Thematik fragen? (Betreuerin,
Kollegin, fachfremde Personen,\ldots{})\strut
\end{minipage}\tabularnewline
\begin{minipage}[t]{0.27\columnwidth}\raggedleft\strut
\strut
\end{minipage} & \begin{minipage}[t]{0.67\columnwidth}\raggedright\strut
Wen könnte ich außerhalb meines Bekanntenkreises fragen, anschreiben,
interviewen?\strut
\end{minipage}\tabularnewline
\begin{minipage}[t]{0.27\columnwidth}\raggedleft\strut
\strut
\end{minipage} & \begin{minipage}[t]{0.67\columnwidth}\raggedright\strut
Welche Personen haben sich zu meiner Problematik bereits
profiliert?\strut
\end{minipage}\tabularnewline
\begin{minipage}[t]{0.27\columnwidth}\raggedleft\strut
\strut
\end{minipage} & \begin{minipage}[t]{0.67\columnwidth}\raggedright\strut
Wer vertritt (im wissenschaftlichen Diskurs, in einer aktuellen
Kontroverse) welche Position?\strut
\end{minipage}\tabularnewline
\begin{minipage}[t]{0.27\columnwidth}\raggedleft\strut
\strut
\end{minipage} & \begin{minipage}[t]{0.67\columnwidth}\raggedright\strut
Gibt es eine Meinungsführerin?\strut
\end{minipage}\tabularnewline
\begin{minipage}[t]{0.27\columnwidth}\raggedleft\strut
\strut
\end{minipage} & \begin{minipage}[t]{0.67\columnwidth}\raggedright\strut
Welche Arbeits- und Forschungsgruppen arbeiten an meinem Thema?\strut
\end{minipage}\tabularnewline
\begin{minipage}[t]{0.27\columnwidth}\raggedleft\strut
\strut
\end{minipage} & \begin{minipage}[t]{0.67\columnwidth}\raggedright\strut
Welche Institution (Institut, Labor, Forschungseinrichtung) könnte ich
besuchen (evt. virtuell über die Homepage im Internet)?\strut
\end{minipage}\tabularnewline
\begin{minipage}[t]{0.27\columnwidth}\raggedleft\strut
\strut
\end{minipage} & \begin{minipage}[t]{0.67\columnwidth}\raggedright\strut
Wie könnte ich kompetente Personen motivieren, für mich ihre kostbare
Zeit zu opfern?\strut
\end{minipage}\tabularnewline
\begin{minipage}[t]{0.27\columnwidth}\raggedleft\strut
\strut
\end{minipage} & \begin{minipage}[t]{0.67\columnwidth}\raggedright\strut
Gibt es dazu eine gute (passende) Gelegenheit (Konferenz, Vortrag an
meiner Uni etc.)?\strut
\end{minipage}\tabularnewline
\bottomrule
\end{longtable}

Nach dieser breit angelegten Erschließung des Themas, bei der alle
gefundenen Hinweise und Informationen als potenziell gleich wichtig
behandelt werden, geht es in der nächsten Phase darum, die eigentliche
Fragestellung herauszuarbeiten. Das vorhandene Material muss nun auf
seine Tauglichkeit für eine interessante Fragestellung untersucht
werden. Dazu ist es notwendig, von den Quellen und Materialien wieder
etwas Distanz zu gewinnen und die eigenen Notizen und Skizzen aus einer
Vogelperspektive zu bewerten. Es geht nicht darum, einzelne Positionen,
Meinungen etc. einzuschätzen, sondern einen Zugang zu den
stichprobenartig gesammelten Materialien zu gewinnen. Dieses Material
wird jetzt ganz gezielt unter dem Gesichtspunkt des Themas angeschaut,
das man bearbeiten möchte:

\begin{longtable}[]{@{}l@{}}
\caption{\textbf{Prüfliste: Sichtung des Materials}}\tabularnewline
\toprule
* Wie viele Literaturhinweise habe ich bei der ersten Suche
gefunden?\tabularnewline
* Ist die Literatur für mich in der verfügbaren Zeitspanne
zugänglich?\tabularnewline
* Wird das Thema in der neuesten Literatur behandelt, gibt es eine
aktuelle Diskussion dazu?\tabularnewline
* Was sind die wichtigsten Standpunkte, Theorien, Beteiligten in dieser
Diskussion?\tabularnewline
* Welche Aspekte werden unbefriedigend behandelt, zu wenig beachtet,
fehlen überhaupt?\tabularnewline
* Was fordert Kritik oder Neugier heraus?\tabularnewline
* Welche Grundlagen und Voraussetzungen habe ich (bzw. fehlen mir), um
das Thema zu bearbeiten?\tabularnewline
* Welche Frage würde ich gern lösen? Welche kann ich
lösen?\tabularnewline
* Welche Sichtweise, Voraussetzungen, Erfahrungen, Zugänge etc. habe
ich, die mich dazu besonders befähigen, diese Frage(n) zu
behandeln?\tabularnewline
\bottomrule
\end{longtable}

Auch bei diesem Schritt hilft es sehr, alle Antworten und Überlegungen
zu notieren, sei es als „Roh-Exposé`` oder bloß in Stichworten, als
Listen, Diagramme oder Mind-Maps. Die Antworten auf diese Fragen können
auf Probleme hinweisen, die die Bearbeitung des Themas behindern oder
sogar unmöglich machen können. Man sollte sie ernst nehmen und letztlich
bei der Entscheidung für oder gegen ein Thema sorgfältig abwägen, welche
Schwierigkeiten damit verbunden sind, welche man in Kauf nehmen will
(oder muss), und welche man -- z. B. durch Eingrenzung oder Abänderung
des Themas -- umgehen kann.

Aktionstabelle: Bewertung des Materials Sie stellen fest, dass \ldots{}
Das kann bedeuten: Tipps: es sehr viel Literatur zu Ihrem Thema gibt •
Das Thema ist zu breit und zu allgemein • Es handelt sich um ein
„Modethema`` • Thema eingrenzen, einen Aspekt herausgreifen •
Entscheidung treffen: Thema ändern oder nicht? es fast keine, auf jeden
Fall zu wenig Literatur zu Ihrem Thema gibt • Das Thema ist sehr eng und
speziell • Das Thema ist neu und noch wenig „beackert``, viel
eigenständiges Arbeiten notwendig • Thema etwas weiter und allgemeiner
fassen • Entscheidung treffen: Fähigkeit, Zeit und Bereitschaft zu
großem Arbeitsaufwand vorhanden und gerechtfertigt? kaum neuere,
aktuelle Literatur zu finden ist • Das Thema gilt in der Fachwelt als
veraltet, ausreichend behandelt • Prüfen: gibt es genügend (neue)
Aspekte und Gründe, um das Thema wieder aufzugreifen? es viele
Institutionen, laufende Projekte, Diskussionen, Veranstaltungen etc. zum
Thema gibt • Modethema • viel aktuelles Material zu bearbeiten,
möglicherweise wenig „gesicherte`` Grundlagen, keine „Standardwerke`` •
laufende Materialvermehrung während der Arbeit • s. oben zu „Modethema``
• strenge Qualitätsprüfung der zu berücksichtigenden Materialien
notwendig

• festen „Stichtag`` für den Abschluss der Materialsammlung vorsehen die
meisten Beiträge zu Ihrem Thema aus anderen Disziplinen stammen •
„Ihre`` Disziplin beschäftigt sich (noch) zu wenig mit dem Thema • Das
Thema lässt sich innerhalb Ihrer Disziplin nicht befriedigend behandeln
(lt. Fachmeinung) • evt. Fragestellung auf möglichen Beitrag Ihrer
Disziplin ausrichten • Prüfen: fundierte Argumente, die diese Meinung
widerlegen? Lohnt sich der Aufwand, den Gegenbeweis anzutreten? die
meisten Beiträge zu Ihrem Thema aus anderen Ländern stammen,
fremdsprachig erschienen sind, etc. • Das Thema wird in Ihrem
Sprachraum/Kulturkreis (noch) zu wenig behandelt •
Fremdsprachenkenntnisse sind Voraussetzung für die Bearbeitung des
Themas • evt. Ansatzpunkt für speziell auf diese Lücke ausgerichtete
Fragestellung • Vorkenntnisse sollten vorhanden sein: Aneignung meist zu
zeitaufwendig, nur für längerfristige Beschäftigung mit dem Thema
sinnvoll alle Fragen, die Sie haben, bereits behandelt wurden • Das
Thema bietet keine neuen Aspekte und Lücken an • Sie haben in
eingefahrenen Bahnen gedacht • Für manche Arbeiten genügt es,
Vorhandenes aufzuarbeiten • Führen Sie Gespräche über das Thema;
schaffen Sie sich Abwechslung Ihnen Vorwissen, Fertigkeiten, Grundlagen
etc. fehlen • Das Thema ist für Ihren Stand zu „hoch`` • Bearbeitung des
Themas kann zeitaufwendig werden • Machen Sie es sich nicht schwerer,
als gefordert ist • Prüfen: Lohnt sich die Aneignung der fehlenden
Grundlagen (evt. in Hinblick auf spätere Weiterarbeit am Thema)? Ist sie
im vorgegebenen oder geplanten Zeitraum möglich?

\subsubsection{Thematik eingrenzen}\label{thematik-eingrenzen}

Schon mehrmals war davon die Rede, dass ein Thema eingegrenzt werden
muss, um mit einem vertretbaren Aufwand bearbeitet werden zu können. Zu
breit und unspezifisch angelegte Themen sind das häufigste und
schwerwiegendste Problem bei wissenschaftlichen Qualifikationsarbeiten!
Wenn dieses Problem nicht rechtzeitig erkannt und behandelt wird, kann
das zu „Jahrhundertprojekten`` führen, die entweder um ein Vielfaches
mehr an Zeit und Aufwand benötigen, als je geplant war, oder überhaupt
nie zum Abschluss kommen.

Während das „Öffnen`` des Themas erfahrungsgemäß relativ wenig
Schwierigkeiten bereitet, ist die gegenläufige Aktivität, d. h. die
Einschränkung und Spezifizierung der Fragestellung, eine komplizierte
und schwer zu vermittelnde Fertigkeit beim wissenschaftlichen Arbeiten.
Es gilt die Faustregel: Je allgemeiner das Thema, desto oberflächlicher
die Fragestellung. Wenn das nachfolgende Beispiel vielleicht auch ein
wenig überzeichnet ist, so ist es in der Tendenz typisch: Anfänglich
wollte ein Student über „Das Weltbevölkerungsproblem`` dissertieren --
letztendlich schrieb er eine Arbeit über „Die Entwicklung der
Überbevölkerung in Mexiko Stadt zwischen 1950-70 in der Sicht
zeitgenössischer marxistischer Literater aus Lateinamerika``.

Die oft langen und komplizierten Titel wissenschaftlicher Arbeiten
entstehen gerade dadurch, dass die Eingrenzung des Themas und seiner
Behandlung darin wiedergegeben werden. Das ist zwar einerseits
„korrekt``, weil es bei der Leserin (bzw. Begutachterin) keine falschen
und zu hohen Erwartungen weckt, aber andererseits nicht sehr
lesefreundlich. Es ist nicht unbedingt notwendig, dass der endgültige
Titel alle gemachten Einschränkungen wiedergibt. Aber es darf nicht
versäumt werden, die Einschränkung des Themas in der Arbeit genau
darzustellen, und in manchen Fällen auch zu begründen. Was Sie brauchen,
ist ein „griffiger`` d. h. interessanter Arbeitstitel, den sie (im
Untertitel, in der Einleitung etc.) einschränken bzw. spezifizieren.

Aktionstabelle: Thema eingrenzen Sie beschränken sich auf \ldots{} Sie
bearbeiten daher \ldots{} Beispiele einen Zeitraum eine bestimmte
Periode in der Geschichte einen bestimmten Abschnitt im Werk eines
Autors \ldots{} in der Zwischenkriegszeit \ldots{} des frühen
Wittgenstein einen geographischen Raum nur ein bestimmtes Land (oder
sogar Region), evt. zwei Länder in Gegenüberstellung \ldots{} in
Mitteleuropa \ldots{} in Kroatien \ldots{} in Oberkärnten bestimmte
Quellen bestimmte Autoren ein Werk oder eine Gruppe von Werken eines
Autors einen Typ von Quellen ein Medium \ldots{} im Briefverkehr
zwischen Freud und Jung \ldots{} in österreichischen Schulbüchern
\ldots{} im ORF eine Betrachtungsweise einen theoretischen Ansatz eine
Disziplin bzw. Teilgebiet der Disziplin eine Ebene, z. B.
methodologisch, \ldots{} \ldots{} innerhalb des marxistischen Paradigmas
\ldots{} der Sozialpsycholgie \ldots{} in Giddens' Strukturationstheorie
einen Einzelfall, einen Anwendungsbereich, z. B. Biographie, Fallstudie,
Institution, soziale Gruppe, Ereignisse \ldots{} \ldots{}Sabine
Spielrein und ihre Beziehung zu Freud und Jung wenige Parameter/
Faktoren eines Systems einzelne Aspekte und betrachten die übrigen
Parameter als konstant, unabhängig („ceteris paribus``-Annahme) zwei
Faktoren, Aspekte etc. in ihrer Beziehung zueinander
(``und``-Verknüpfung) \ldots{} unter der Annahme eines stagnierenden
Bevölkerungswachstums \ldots{} Bevölkerungspolitk und individuelle
Familienplanung auf einen inhaltlichen Aspekt ein (typisches) Phänomen
oder Problem, das Sie herausgreifen \ldots{} Angstzustände in
Zusammenhang mit dem Aidstest

\subsection{Ein Konzept erstellen}\label{ein-konzept-erstellen}

Den Abschluss dieser ersten Phase stellt das Konzept oder Exposé dar.
Ein (genehmigtes) Konzept ist das Endprodukt der Orientierungs- und
Planungsphase. Es ist Grundlage, aber auch Startschuss für die
eigentliche Arbeit. Selbst wenn es nicht immer formal erforderlich ist,
so ist ein schriftliches Exposé doch eine sinnvolle Sache: Es ist die
Zusammenfassung und der Abschluss der Vorbereitungsphase, darüber hinaus
aber auch eine erste Schreibübung, mit der man sich am Thema versucht,
und ein Leitfaden, zu dem man später immer wieder zurückkehren kann und
soll.

Im Gegensatz zu einem formellen Projektantrag -- der im Aufbau ganz
ähnlich wäre und die selben Fragen beantworten müsste -- braucht das
Exposé keine endgültige Festlegung der Arbeit zu sein. Es kann -- und
wird -- sich in vielen Fällen noch ändern, z. B. weil das Thema doch
noch weiter eingeschränkt werden muss, weil sich durch die Einarbeitung
ins Thema Ihr Wissensstand und Ihre Interessen ändern, usw. Es ist
vielmehr eine Momentaufnahme des Wissens und der Fragestellung, die aber
nicht an einem beliebigen Moment entstanden ist, sondern genau zu dem
Zeitpunkt, wo Sie sich in das gründliche Recherchieren, Lesen und
Bearbeiten des Materials stürzen -- also gerade bevor die Gefahr des
„Sich-Verzettelns`` akut wird. Das Exposé hilft dabei, diese Gefahr zu
vermeiden, wenn man es immer wieder hervorholt und nicht leichtfertig
verwirft.

An vielen Fachbereichen bzw. Instituten gibt es Merkblätter für die
Erstellung des Konzepts. Es ist schwierig, allgemeine Richtlinien zu
geben, was in ein Exposé alles -- und in welchem Detaillierungsgrad --
hineingehört. Je nach Art und Umfang der Arbeit (z. B. Literaturarbeit
vs.~empirische Arbeit, kleinere Seminararbeit vs.~Buch,
Qualifikationsarbeit vs.~Projekteinreichung usw.) gibt es
unterschiedliche Voraussetzungen und Anforderungen an ein Konzept. Die
nachfolgende Liste ist daher bloß eine allgemeine Orientierungshilfe und
entsprechend den Gegebenheiten abzuwandeln:

Prüfliste: Bestandteile eines Exposés * Was ist die Ausgangslage, der
Stand der Forschung? Welche Erkenntnisse liegen bereits vor?
(Beschreibung des „state of the art``) * Was ist mein
Erkenntnisinteresse, meine Motivation für diese Arbeit? * Welches
(theoretische, praktische, empirische, soziale, politische,\ldots{})
Problem ist der Ausgangspunkt meiner Arbeit? * Zu welchem Ziel
(Ergebnis) soll meine Arbeit führen? * Welche Fragen will ich in meiner
Arbeit behandeln (Thema)? * Welche Fragen bzw. Aspekte der Fragen werde
ich in meiner Arbeit NICHT behandeln? (Eingrenzung des Themas) * Welche
Quellen will ich dazu verwenden? (nur im Überblick, noch keine
detaillierte Bibliografie -- die gibt es ja noch nicht!) * Welche
Quellen werde ich NICHT verwenden? (Eingrenzung des Materials) * Welche
Methode will ich anwenden? * Welche Hilfe bzw. Hilfsmittel (Beratung,
Infrastruktur, Software, Reisemittel,\ldots{}) benötige ich für meine
Arbeit? * Wie könnte eine vorläufige Gliederung meiner Arbeit aussehen?
* Welche Arbeitsschritte, Phasen, Zwischenergebnisse plane ich? * Wie
könnte ein grober Zeitplan aussehen? Bis wann will ich welche Etappen
der Arbeit abgeschlossen haben?

\subsubsection{Grünes Licht?}\label{grunes-licht}

Das Exposé ist auch eine konkrete Grundlage, mit der man die geplante
Arbeit mit einer Betreuerin, Projektleiterin, Projektteam etc.
absprechen kann. Dieses Gespräch hat mehrere Ziele:

\begin{itemize}
\tightlist
\item
  Hilfe und Ratschläge einholen
\item
  Probleme des Konzepts, va. seine Machbarkeit klären
\item
  „grünes Licht`` für den Arbeitsbeginn bekommen
\end{itemize}

Im Idealfall übergeben Sie Ihren Gesprächspartnern das Papier ein paar
Tage vor dem Gesprächstermin, um ihm oder ihr Zeit zu lassen, es
gründlich zu lesen und sich Gedanken dazu zu machen. Bereiten Sie dieses
Gespräch nicht nur umsichtig vor, sondern nehmen Sie sich auch die Zeit,
es gründlich nachzubereiten.

Prüfliste: Betreuungsgespräch zum Exposé * VOR dem Gespräch: Wie könnte
ich mein Konzept in einem mündlichen Gespräch kurz darstellen? (sinnvoll
als Einleitung, auch wenn Sie annehmen können, dass Ihre
Gesprächspartnerin den Inhalt des Exposés kennt) * Welche Schwachpunkte,
offenen Fragen, Unklarheiten möchte ich ansprechen? (Sie sind nicht in
einer Prüfungssituation: Weisen Sie selbst auf Probleme in Ihrem Konzept
hin, zu denen Sie Tipps und Ratschläge brauchen) * Was will ich im
Gespräch erreichen? („grünes Licht``, Literaturhinweise, Kontakte, einen
weiteren Termin usw.) * WÄHREND des Gesprächs: Welche Hinweise,
Anmerkungen und/oder Kritiken zu meinem Konzept haben sich im Laufe des
Gesprächs ergeben? * Welche dieser Hinweise, Anmerkungen, Kritiken sind
leicht in mein Konzept einzubauen? * Welche Hinweise, Anmerkungen,
Kritiken laufen auf eine komplette Neubearbeitung hinaus? * Wie lässt
sich das gemeinsam erreichte Verständnis zusammenfassen? (Versuchen Sie
bereits am Ende des Gesprächs eine Zusammenfassung zu geben, die das
gemeinsam erreichte Verständnis noch einmal absichert.) * Was sind meine
nächsten Arbeitsschritte? (Vereinbaren Sie wenn möglich gleich einen
weiteren Gesprächstermin zu einem konkreten Inhalt, z. B. umgearbeitetes
Konzept, Literaturliste) * Soll eine schriftliche Zusammenfassung der
Ergebnisse des Gesprächs an die Betreuerin geschickt werden?

Notieren Sie sich Hinweise (z. B. auf Literatur, Materialien, Namen
etc.) um darauf später zurückkommen zu können, sowie Anmerkungen und
Kritiken, um später genauer beurteilen zu können, welche Veränderungen
sie für Ihr Konzept bedeuten. Nehmen Sie Hinweise, die auf grundlegende
Änderungen des Konzepts hinauslaufen, ernst, aber nehmen Sie sich auch
Zeit, darüber nachzudenken -- evt. bis zu einem weiteren
Gesprächstermin. Sie müssen im Gespräch nicht gleich ein Gegenargument
bei der Hand haben -- entweder es fällt Ihnen später ein, dann ist es
vielleicht ein Hinweis auf eine Argumentation, die Sie in der Arbeit
behandeln könnten. Oder es fällt Ihnen keines ein -- dann ist es ein
Hinweis auf eine Schwäche im Konzept.

Fassen Sie auf jeden Fall für sich selbst die Ergebnisse schriftlich
zusammen. In manchen Fällen (z. B. überarbeiteter, „vergesslicher``
Betreuer) kann ein übermitteltes schriftliches Protokoll für Sie einen
Schutz darstellen (``Sicherung des Erreichten``). Klären Sie das aber
vorher noch mündlich ab, weil Ihre Betreuerin vielleicht so ein
unverlangtes formales Verhalten, wie es z. B. ein Protokoll darstellt,
ablehnt.

Entscheidend für Ihre weitere Vorgangsweise ist die grundlegende und mit
der Betreuerin gemeinsam erreichte Beurteilung des Betreuungsgespräches:
Gab es eine „Stop`` oder „Go``-Vereinbarung? Müssen Sie Ihr Konzept
nochmals überarbeiten, oder können Sie -- evt. mit kleinen Änderungen --
mit der eigentlichen Arbeit beginnen? Aufgabe: 1. Suchen Sie ein Thema
freier Wahl für eine wissenschaftliche Arbeit.. Finden Sie einen
(vorläufigen) Titel für die Arbeit. 2. Erschließen Sie das Thema mit
einer Methode Ihrer Wahl (freies Schreiben, Assoziieren, Mindmapping
\ldots{}) 3. Erläutern Sie kurz (auf der Grundlage Ihrer ersten Notizen
bzw. Mindmap, Liste \ldots{} ) in wenigen Sätzen, und noch ohne gezielte
Recherchen durchgeführt zu haben Ihre persönliche Motivation, Ihr
Forschungsinteresse, Ihre Fragestellung(en), warum Sie dieses Thema
beschäftigt. 4. Nun grenzen Sie das Thema auf drei verschiedene Arten
ein (s. Aktionstabelle „Thema eingrenzen``) und geben dies durch
Untertitel wieder. Geben Sie zu jedem eingegrenzten Thema Ihre
Fragestellung an. (Vernachlässigen Sie dabei fürs erste die
Realisierbarkeit des Themas)


\end{document}
