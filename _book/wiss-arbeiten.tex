\documentclass[]{book}
\usepackage{lmodern}
\usepackage{amssymb,amsmath}
\usepackage{ifxetex,ifluatex}
\usepackage{fixltx2e} % provides \textsubscript
\ifnum 0\ifxetex 1\fi\ifluatex 1\fi=0 % if pdftex
  \usepackage[T1]{fontenc}
  \usepackage[utf8]{inputenc}
\else % if luatex or xelatex
  \ifxetex
    \usepackage{mathspec}
  \else
    \usepackage{fontspec}
  \fi
  \defaultfontfeatures{Ligatures=TeX,Scale=MatchLowercase}
\fi
% use upquote if available, for straight quotes in verbatim environments
\IfFileExists{upquote.sty}{\usepackage{upquote}}{}
% use microtype if available
\IfFileExists{microtype.sty}{%
\usepackage{microtype}
\UseMicrotypeSet[protrusion]{basicmath} % disable protrusion for tt fonts
}{}
\usepackage[margin=1in]{geometry}
\usepackage{hyperref}
\hypersetup{unicode=true,
            pdftitle={Wissenschaftliches Arbeiten im digitalen Zeitalter},
            pdfauthor={Peter Baumgartner},
            pdfborder={0 0 0},
            breaklinks=true}
\urlstyle{same}  % don't use monospace font for urls
\usepackage{natbib}
\bibliographystyle{apalike}
\usepackage{color}
\usepackage{fancyvrb}
\newcommand{\VerbBar}{|}
\newcommand{\VERB}{\Verb[commandchars=\\\{\}]}
\DefineVerbatimEnvironment{Highlighting}{Verbatim}{commandchars=\\\{\}}
% Add ',fontsize=\small' for more characters per line
\usepackage{framed}
\definecolor{shadecolor}{RGB}{248,248,248}
\newenvironment{Shaded}{\begin{snugshade}}{\end{snugshade}}
\newcommand{\KeywordTok}[1]{\textcolor[rgb]{0.13,0.29,0.53}{\textbf{#1}}}
\newcommand{\DataTypeTok}[1]{\textcolor[rgb]{0.13,0.29,0.53}{#1}}
\newcommand{\DecValTok}[1]{\textcolor[rgb]{0.00,0.00,0.81}{#1}}
\newcommand{\BaseNTok}[1]{\textcolor[rgb]{0.00,0.00,0.81}{#1}}
\newcommand{\FloatTok}[1]{\textcolor[rgb]{0.00,0.00,0.81}{#1}}
\newcommand{\ConstantTok}[1]{\textcolor[rgb]{0.00,0.00,0.00}{#1}}
\newcommand{\CharTok}[1]{\textcolor[rgb]{0.31,0.60,0.02}{#1}}
\newcommand{\SpecialCharTok}[1]{\textcolor[rgb]{0.00,0.00,0.00}{#1}}
\newcommand{\StringTok}[1]{\textcolor[rgb]{0.31,0.60,0.02}{#1}}
\newcommand{\VerbatimStringTok}[1]{\textcolor[rgb]{0.31,0.60,0.02}{#1}}
\newcommand{\SpecialStringTok}[1]{\textcolor[rgb]{0.31,0.60,0.02}{#1}}
\newcommand{\ImportTok}[1]{#1}
\newcommand{\CommentTok}[1]{\textcolor[rgb]{0.56,0.35,0.01}{\textit{#1}}}
\newcommand{\DocumentationTok}[1]{\textcolor[rgb]{0.56,0.35,0.01}{\textbf{\textit{#1}}}}
\newcommand{\AnnotationTok}[1]{\textcolor[rgb]{0.56,0.35,0.01}{\textbf{\textit{#1}}}}
\newcommand{\CommentVarTok}[1]{\textcolor[rgb]{0.56,0.35,0.01}{\textbf{\textit{#1}}}}
\newcommand{\OtherTok}[1]{\textcolor[rgb]{0.56,0.35,0.01}{#1}}
\newcommand{\FunctionTok}[1]{\textcolor[rgb]{0.00,0.00,0.00}{#1}}
\newcommand{\VariableTok}[1]{\textcolor[rgb]{0.00,0.00,0.00}{#1}}
\newcommand{\ControlFlowTok}[1]{\textcolor[rgb]{0.13,0.29,0.53}{\textbf{#1}}}
\newcommand{\OperatorTok}[1]{\textcolor[rgb]{0.81,0.36,0.00}{\textbf{#1}}}
\newcommand{\BuiltInTok}[1]{#1}
\newcommand{\ExtensionTok}[1]{#1}
\newcommand{\PreprocessorTok}[1]{\textcolor[rgb]{0.56,0.35,0.01}{\textit{#1}}}
\newcommand{\AttributeTok}[1]{\textcolor[rgb]{0.77,0.63,0.00}{#1}}
\newcommand{\RegionMarkerTok}[1]{#1}
\newcommand{\InformationTok}[1]{\textcolor[rgb]{0.56,0.35,0.01}{\textbf{\textit{#1}}}}
\newcommand{\WarningTok}[1]{\textcolor[rgb]{0.56,0.35,0.01}{\textbf{\textit{#1}}}}
\newcommand{\AlertTok}[1]{\textcolor[rgb]{0.94,0.16,0.16}{#1}}
\newcommand{\ErrorTok}[1]{\textcolor[rgb]{0.64,0.00,0.00}{\textbf{#1}}}
\newcommand{\NormalTok}[1]{#1}
\usepackage{longtable,booktabs}
\usepackage{graphicx,grffile}
\makeatletter
\def\maxwidth{\ifdim\Gin@nat@width>\linewidth\linewidth\else\Gin@nat@width\fi}
\def\maxheight{\ifdim\Gin@nat@height>\textheight\textheight\else\Gin@nat@height\fi}
\makeatother
% Scale images if necessary, so that they will not overflow the page
% margins by default, and it is still possible to overwrite the defaults
% using explicit options in \includegraphics[width, height, ...]{}
\setkeys{Gin}{width=\maxwidth,height=\maxheight,keepaspectratio}
\IfFileExists{parskip.sty}{%
\usepackage{parskip}
}{% else
\setlength{\parindent}{0pt}
\setlength{\parskip}{6pt plus 2pt minus 1pt}
}
\setlength{\emergencystretch}{3em}  % prevent overfull lines
\providecommand{\tightlist}{%
  \setlength{\itemsep}{0pt}\setlength{\parskip}{0pt}}
\setcounter{secnumdepth}{5}
% Redefines (sub)paragraphs to behave more like sections
\ifx\paragraph\undefined\else
\let\oldparagraph\paragraph
\renewcommand{\paragraph}[1]{\oldparagraph{#1}\mbox{}}
\fi
\ifx\subparagraph\undefined\else
\let\oldsubparagraph\subparagraph
\renewcommand{\subparagraph}[1]{\oldsubparagraph{#1}\mbox{}}
\fi

%%% Use protect on footnotes to avoid problems with footnotes in titles
\let\rmarkdownfootnote\footnote%
\def\footnote{\protect\rmarkdownfootnote}

%%% Change title format to be more compact
\usepackage{titling}

% Create subtitle command for use in maketitle
\newcommand{\subtitle}[1]{
  \posttitle{
    \begin{center}\large#1\end{center}
    }
}

\setlength{\droptitle}{-2em}
  \title{Wissenschaftliches Arbeiten im digitalen Zeitalter}
  \pretitle{\vspace{\droptitle}\centering\huge}
  \posttitle{\par}
  \author{Peter Baumgartner}
  \preauthor{\centering\large\emph}
  \postauthor{\par}
  \predate{\centering\large\emph}
  \postdate{\par}
  \date{2017-10-07}

\usepackage{booktabs}
\usepackage[english, ngerman]{babel}

\usepackage{amsthm}
\newtheorem{theorem}{Theorem}[chapter]
\newtheorem{lemma}{Lemma}[chapter]
\theoremstyle{definition}
\newtheorem{definition}{Definition}[chapter]
\newtheorem{corollary}{Corollary}[chapter]
\newtheorem{proposition}{Proposition}[chapter]
\theoremstyle{definition}
\newtheorem{example}{Example}[chapter]
\theoremstyle{definition}
\newtheorem{exercise}{Exercise}[chapter]
\theoremstyle{remark}
\newtheorem*{remark}{Remark}
\newtheorem*{solution}{Solution}
\begin{document}
\maketitle

{
\setcounter{tocdepth}{2}
\tableofcontents
}
\chapter*{Vorwort}\label{vorwort}
\addcontentsline{toc}{chapter}{Vorwort}

This is a \emph{sample} book written in \textbf{Markdown}. You can use
anything that Pandoc's Markdown supports, e.g., a math equation
\(a^2 + b^2 = c^2\).

The \textbf{bookdown} package can be installed from CRAN or Github:

\begin{Shaded}
\begin{Highlighting}[]
\KeywordTok{install.packages}\NormalTok{(}\StringTok{"bookdown"}\NormalTok{)}
\CommentTok{# or the development version}
\CommentTok{# devtools::install_github("rstudio/bookdown")}
\end{Highlighting}
\end{Shaded}

Remember each Rmd file contains one and only one chapter, and a chapter
is defined by the first-level heading \texttt{\#}.

To compile this example to PDF, you need to install XeLaTeX.

\chapter*{Einleitung}\label{einleitung}
\addcontentsline{toc}{chapter}{Einleitung}

Placeholder

\chapter{Konzipieren einer Arbeit}\label{konzipieren-einer-arbeit}

Placeholder

\section{Von der Idee zum Konzept}\label{von-der-idee-zum-konzept}

\section{Rahmenbedingungen klären}\label{rahmenbedingungen-klaren}

\subsection{Zeitplan für eine größere Arbeit (z. B.
Diplomarbeit)}\label{zeitplan-fur-eine-groere-arbeit-z.-b.-diplomarbeit}

\subsection{Zeitplan für eine
Semesterarbeit}\label{zeitplan-fur-eine-semesterarbeit}

\section{Eine Fragestellung
erarbeiten}\label{eine-fragestellung-erarbeiten}

\subsection{Thematik erschließen,
„öffnen``}\label{thematik-erschlieen-offnen}

\subsection{Material sammeln}\label{material-sammeln}

\subsection{Material sichten und
bewerten}\label{material-sichten-und-bewerten}

\subsection{Thematik eingrenzen}\label{thematik-eingrenzen}

\section{Ein Konzept erstellen}\label{ein-konzept-erstellen}

\subsection{Grünes Licht?}\label{grunes-licht}

\chapter{Recherchieren}\label{recherchieren}

Placeholder

\section{Material sammeln}\label{material-sammeln-1}

\subsection{Überblick}\label{uberblick}

\subsection{Kreatives Recherchieren}\label{kreatives-recherchieren}

\section{Beginn der Literatursuche}\label{beginn-der-literatursuche}

\section{Systematische Suche}\label{systematische-suche}

\section{Literaturliste erstellen}\label{literaturliste-erstellen}

\subsection{Software für
Literaturverwaltung}\label{software-fur-literaturverwaltung}

\section{Literatur prüfen und auswählen -- Leseliste
erstellen}\label{literatur-prufen-und-auswahlen-leseliste-erstellen}

\section{Litertur beschaffen}\label{litertur-beschaffen}

\subsection{Bücher beschaffen}\label{bucher-beschaffen}

\subsection{Unselbständige Werke
beschaffen}\label{unselbstandige-werke-beschaffen}

\section{Material ordnen}\label{material-ordnen}

\chapter{Methods}\label{methods}

Placeholder

\chapter{Applications}\label{applications}

Placeholder

\section{Example one}\label{example-one}

\section{Example two}\label{example-two}

\chapter{Final Words}\label{final-words}

Placeholder

\chapter{Placeholder}\label{placeholder}

\bibliography{packages.bib,book.bib}


\end{document}
